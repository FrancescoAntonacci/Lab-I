\documentclass{article}
\usepackage{graphicx}
\usepackage{amsmath}
\usepackage{float}
\title{Misurazione della refrattività del'acqua}
\author{Alessandro Di Meglio \\ Francesco Angelo Fabiano Antonacci}
\date{\today}
\begin{document}

\maketitle
\section{Scopo dell'esperienza}

Lo scopo dell'esperienza è quello di misurare la refrattività  dell'acqua mediante una lente cilindrica c.

\section{Cenni teorici}

Data una lente la relazione tra f, ossia la distanza focale della lente, p ,la distanza tra il centro della lente e la sorgente luminosa, e q , la distanza tra il centro della lente e l'immagine messa a fuoco, è la legge dei punti coniugati:

\begin{equation}
\frac{1}{f}=\frac{1}{p}+\frac{1}{q}
\label{pucon}
\end{equation}

Ponendo $\frac{1}{p}=x$ ,  $\frac{1}{q}=y$ e $\frac{1}{f}=c$ otteniamo la seguente relazione:
\begin{equation}
y=c-x
\label{:)}
\end{equation}
 


Data una lente di raggio r costituita da un mezzo con un indice di rifrazione n, l'equazione del costruttore di lenti ci dà una relazione con la distanza focale:

\begin{equation}
\frac{1}{f}=(n-1)\frac{2}{r}(1-\frac{n-1}{n})
\label{lme}
\end{equation}

Da cui otteniamo l'equazione per stimare la refrattività $\eta$:

\begin{equation}
\eta=\frac{r}{2f-r}
\label{refr}
\end{equation}


\section{Apparato strumentale}

\subsection{Materiale Utilizzato}

Per l'esperienza sono stati utilizzati i seguenti strumenti:
\begin{itemize}

\item Schermo
\item Smartphone
\item Bottiglia cilindrica di vetro
\item Nastro adesivo
\item Filo
\end{itemize}


\subsection{Misure di lunghezza}
Per le misure di lunghezza è stato utilizzato un metro a nastro con risoluzione di 1 mm.



\section{Descrizione delle misure}

E' stata costruita la lente riempendo la bottiglia di acqua.
Sono stati compiuti 4 giri di spago attorno alla bottiglia, si è presa la lunghezza dello spago che ha avvolto la bottiglia.
E' stato fissato un metro  a nastro su un banco per poter prendere le coordinate degli oggetti del banco ottico.
E' stata posizionata la lente a una coordinata che è rimasta fissa per tutto l'esperimento.
E' stata attivata la torcia dello smartphone.
E' stato posizionato lo smartphone in successive posizioni; in ciascuna è  stata presa la distanza tra il centro della lente e lo schermo nella configurazione in cui la luce proiettata su esso era a fuoco.
E' stata dedicata particolare cura a tenere l'asse passante per la torcia e il centro della lente  parallelo al metro a nastro.

\subsection{Incertezze sulle misure di posizione}

Si veda la sezione \textbf{Cenni teorici} per la notazione usata.

\subsection{Incertezza su p}
L'incertezza sulla distanza tra telefono e centro della bottiglia è stata assunta 1mm a causa della risoluzione del metro a nastro.
E' stato verificato che l'effetto del possibile disallineamento (si veda:\textbf{Descrizione delle misure}) sulla misura di p influisce per meno di un decimo dell'incertezza dovuta alla risoluzione del metro a nastro. 

\subsection{Incertezza su q}

Per stimare l'incertezza su q è necessario tenere in considerazione il fatto che la cofigurazione in cui la lente è a fuoco avviene in un intorno di un punto.
Prendendo ripetute volte le misure si è osservato che questo intervallo arriva a essere ampio fino a 4 millimetri.
Come incertezza sarà presa la somma in quadratura tra la deviazione standard della distribuzione uniforme su questo intervallo e la risolizione strumentale del metro a nastro.
Come nel caso precedente è stato verificato che l'effetto del possibile disallineamento sulla misura di q influisce per meno di un decimo dell'incertezza dovuta all'incertezza stimata.

\subsection{Incertezza su r}
L'incertezza sul raggio è l'incertezza sulla misura di 4 volte la circonferenza diviso $8\pi$.
E' ragionevole assumere che l'incertezza sulla misura della circonferenza sia la somma in quadratura dello spessore dei quattro giri e della risoluzione del metro a nastro.
Pertanto $\sigma r= 0.1$[mm].

\subsection{Indipendenza delle variabili}

\section{Analisi dei dati}
\subsection{Algoritmo di best fit}
Per trovare la distanza focale verrà fatto un fit per la relazione (\ref{:)}) ; sarà aggiunto un parametro m che si desidera essere compatibile con -1.
La legge per la quale si fa il fit sarà:
\begin{equation}
	y=mx+c
	\label{fiteq}
\end{equation}


		\begin{table}
			\centering
				\begin{tabular}{|c|c|c|}

					\hline
						$m$  & $c$[m] & $f$[m]\\
					\hline
			
						$-1.01\pm0.05$ & $11.4\pm0.2$ & $0.088\pm0.001$\\
					\hline
			
				\end{tabular}
					\caption{ Parametri ricavati dall'algoritmo di best-fit}
		
		\end{table}		


\subsection{Test del $ X^2$ e p-value}
Il valore dei gradi di libertà è $\nu=6$: abbiamo campionato 8 punti e abbiamo utilizzato 2 parametri nel nostro modello. 
Il chi-quadro stimato dall'algoritmo di best-fit è $X^2=6$.
Il p-value corrispondente è $p=0.88$.


\subsection{Misura di $\eta$}
Utilizzando l'equazione  ( \ref{refr} ) si ottiene $\eta$. 
In tabella (\ref{etameas} ) sono riportati i risultati.

		\begin{table}
			\centering
				\begin{tabular}{|c|c|c|}

					\hline
						$\eta$  & Contributo a $\sigma \eta$ di $\sigma r$ & Contributo a $\sigma \eta$ di $\sigma f$\\
					\hline
			
						$0.327\pm0.004$ & $ 0.0004$ & $ 0.0035$\\
					\hline
			
				\end{tabular}
					\caption{}
					\label{etameas}
		
		\end{table}		



\section{Conclusione}

\end{document}