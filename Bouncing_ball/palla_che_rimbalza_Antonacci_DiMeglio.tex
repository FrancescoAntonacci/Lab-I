\documentclass{article}
\usepackage{graphicx} % Required for inserting images

\title{Valutazione sperimentale di una legge sulle altezze massime dei rimbalzi di una sfera}
\author{Alessandro Di Meglio \\Francesco Angelo Fabiano Antonacci}
\date{\today}

\begin{document}

\maketitle

\section{Scopo dell'esperienza}
Lo scopo dell'esperienza è quello di valutare l'accuratezza di una legge nel modellizzare l'altezza raggiunta in rimbalzi successivi da una sfera lasciata cadere.


\section{Cenni teorici}
Assumiamo che ciascun rimbalzo assorba la stessa frazione di energia.
L'altezza massima di ogni rimbalzo seguirà la seguente equazione:
\begin{equation}
h_{n} = h_{0}  \gamma^{n}
\label{h(n)}
\end{equation}
Dove $h_0$ è il numero di rimalzi,$ \gamma$ è il rapporto tra due altezza massime successive, n è il numero di rimbalzi. 
Questa è la legge che in questa relazione ci si prefigge di verificare.

Inoltre è possibile determinare l'altezza raggiunta fra 2 rimbalzi consecutivi come segue:

\begin{equation}
 	h_{n} = \frac{1}{8} g[(t_{n} - t_{n-1})^2] 
		\label{h2(n)}
\end{equation}

\section{Apparato strumentale}

\subsection{Misure di tempi}
Le misure di tempi sono state effettuate mediante uno smartphone con risoluzione di un 0.1ms .
Ciononostante, si è stimato che l'incertezza sulle misure dei tempi  $\sigma_t=1.5$ [ms]  fosse la metà del tempo di contatto .
Per l'incertezza sulle differenze dei tempi(dt) in cui sono avventuti gli urti è stata usata la seguente equazione:  $\sigma_{dt}=\sigma_t\sqrt{2}$.


\subsection{Misure di lunghezza}
Le misure di lunghezza sono state prese con un metro a nastro con risoluzione di un millimetro.
Per calcolare le incertezza($\sigma_h$) delle altezze massime stimate(h) è stata usata la seguente equazione:
\begin{equation}
	\sigma_{h} = \frac{\sigma_{dt} h  2 \sqrt{2}   } {dt}
\end{equation}


\subsection{Materiale}
Una biglia di acciaio è stato l'oggetto con cui cercare di verificare la legge.


\section{Descrizione delle misure}

\subsection{Misura dei periodi dei rimbalzi}
E'stata misurata l'altezza del supporto.
E' stata avviata una registrazione audio dei rimbalzi con lo smartphone. 
E' stata lasciata cadere la biglia da un supporto.

\subsection{Misura altezza iniziale}

L'altezza iniziale è stata misurata con un metro a nastro.
La relativa incertezza è stata presa di 5 millimetri anziché di 1 in quanto non è stato garatito un rilascio perfetto.

\centering

		$h_{0} (misurata)=0.373\pm0.005$ m

\raggedright

\section{Analisi dei dati}




\subsection{Algoritmo di best fit}
Il numero di rimbalzi non ha incertezza, pertanto  è stato usato come variabile di incertezza trascurabile per l'algoritmo di best-fit.
La funzione su cui fare il fit utilizzata non è stata (\ref{h(n)}) ma:

\begin{equation}
	h_{n} = h_{0}  \gamma^{n}+c
		\label{h(n)+c}
\end{equation}

Dove c è una costante, utile per scovare un potenziale errore sistematico, che desideriamo sia quanto più vicino a 0  possibile.

\begin{figure}
	\includegraphics[width=\textwidth]{Altezza_dei_rimbalzi.png}
		\caption{Altezze stiamate mediante l'occorrenza degli urti e previsioni di (4) con i parametri dati dall'algoritmo di best-fit}

\end{figure}
\begin{figure}
	\includegraphics[width=\textwidth]{Grafico_dei_residui.png}
		\caption{Grafico dei residui}

\end{figure}


\section{Conclusioni}

\begin{table}
\begin{tabular}{|c|c|c|c|}
\hline
		$h_{0}$ (misura diretta) [m]& $h_{0}$ (algoritmo di best-fit) [m]& $\gamma$ &c (algoritmo di best-fit) [m]\\
\hline
		$0.373\pm0.005$ & $0.364\pm0.009$ & $0.741\pm0.008$ & $0.001\pm0.002$\\
\hline

	
\end{tabular}
	\caption{Risultati dell'algoritmo di best-fit}
\end{table}
Dai risultati dell'algoritmo di best-fit si osserva che: il parametro c è assai inferiore all'incertezza dell'altezza stimata e la sua incertezza fa sì che c=0 sia possibile;l'altezza iniziale misurata e quella stimata sono distanti non più di una barra di errore: le due sono compatibili.
Inoltre  da (Figure 2) si osserva che le previsioni non si discostano dai dati stimati per non più di due barre di errore.
Di conseguenza l'equazione (\ref{h(n)}) è compatibile con quanto è stato misurato.






\end{document}