\documentclass{article}
\usepackage{graphicx} % Required for inserting images
\usepackage{amsmath}
\usepackage{float}
\title{Valutazione sperimentale di diverse oscillazioni accoppiate}
\author{Alessandro Di Meglio \\ Francesco Angelo Fabiano Antonacci}
\date{\today}
\begin{document}

\maketitle

\section{Scopo dell'esperienza} 

Lo scopo dell'esperienza è quello di confrontare e valutare diverse oscillazioni, correlandole fra loro e studiando analogie e differenze.
I tipi di oscillazioni da studiare, con i rispettivi obiettivi da verificare sono:

\begin{itemize}

	    \item OSCILLAZIONE DI UN PENDOLO SINGOLO: confrontare la pulsazione angolare $\omega_{0}$ \\ misurata con la pulsazione angolare teorica
	    \item OSCILLATORE DI UN PENDOLO SINGOLO SMORZATO: medesimo scopo della misura precedente ma disponendo un galleggiante
	    \item OSCILLAZIONE IN FASE: misurazione della pulsazione $\omega_{f}$
	    \item OSCILLAZIONE IN CONTROFASE: misurazione della pulsazione $\omega_{c}$
	    \item BATTIMENTI: misurazione delle pulsazioni angolari portante e modulante da confrontare poi con $\omega_{f}$ e $\omega_{c}$
    
\end{itemize}

\section{Cenni teorici}
 
			\subsection{ Oscillazione di un pendolo singolo}
		\subsection{  Oscillatore di un pendolo singolo smorzato }
		\subsection{  Oscillazioni in fase}
		\subsection{  Oscillazioni in controfase}
		\subsection{  Battimenti}
					    
\section{Apparato strumentale}

Due pendoli sono stati appesi alla stessa altezza grazie a un supporto.

			\begin{itemize}
					   \item galleggiante da pesca 
					   \item molla 
					   \item Metro a nastro con risoluzione di 1 mm 
					   \item calibro Palmer con risoluzione di 0.05 mm
					   \item sistema di presa dati con Arduino
			\end{itemize}

\section{Descrizione delle misure}
	
		\subsection{ Misure dimensioni pendoli }
		\subsection{ Oscillazione di un pendolo singolo}
		\subsection{  Oscillatore di un pendolo singolo smorzato }
		\subsection{  Oscillazioni in fase}
		\subsection{  Oscillazioni in controfase}
		\subsection{  Battimenti}

\section{Stima delle incertezze}
		
		\subsection{Incertezze sulle dimensioni}
		\subsection{Incertezze sulle posizioni}
		\subsection{Incertezze sui tempi}

    
\section{Analisi dei dati}

		\subsection{ Oscillazione di un pendolo singolo}
		\subsection{  Oscillatore di un pendolo singolo smorzato }
		\subsection{  Oscillazioni in fase}
		\subsection{  Oscillazioni in controfase}
		\subsection{  Battimenti}


\section{Conclusioni}
\end{document}