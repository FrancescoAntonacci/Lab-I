\documentclass{article}
\usepackage{graphicx}
\usepackage{amsmath}
\usepackage[margin=2cm]{geometry}
\usepackage{float}
\usepackage{hyperref}

\title{}
\author{}
\date{\today}
\begin{document}

\maketitle
\section{Scopo dell'esperienza}


\section{Cenni teorici}

	\begin{equation}
		2=1+1
			\label{eq:}
	\end{equation}




 

\section{Apparato strumentale}


\subsection{Materiale Utilizzato}


\begin{itemize}

\item .


\end{itemize} 

\subsection{Misure di lunghezza}


\section{Descrizione delle misure}


\section{Analisi dei dati}
 

%\begin{table}[H]
%		\centering
%			\begin{tabular}{|cc|}
%				\hline
%				$c $[m$^{-1}$] & $ -2.9 \pm 0.3$\\
%				$f$[m] & $0.32\pm 0.02$\\
%				$m$ & $ 1.01 \pm 0.04$\\
%				\hline
%			\end{tabular}
%		\caption{Valori ottenuti dall'algoritmo di best-fit per la legge (\ref{eq::)}).}
%		\label{tab:len}	
%\end{table}




%\begin{figure}[H]
%	\includegraphics[width=\textwidth]{Dati_raccolti_residuilen.png}
%	\caption{Grafico dei residui della legge(\ref{eq::)}).}
%	\label{fig:lenres}
%\end{figure}

\subsubsection{Valutazione del modello}



\section{Conclusioni}



\end{document}